\documentclass{article}
\usepackage[margin=1in]{geometry}
\usepackage{amsmath}
\usepackage{bm}
\usepackage{graphicx}

\begin{document}

\begin{titlepage}
    \center
    \textsc{\LARGE AE 4803: Homework 3}\\[1.5cm]
    \textsc{\Large Leader: Madison Stein}\\[0.5cm]
    \textsc{\Large Group Members: Vinh Phuc Bui, Charles Andrew Person, Mahalakshmi Srinivasan, Nathan Wang}\\[2cm]
    \textsc{\large Instructor: Evangelos Theodorou}\\[0.5cm]
    \textsc{\large TA: Zhanzhan Zhao}\\[1cm]
    \textsc{\large November 21, 2018}
\end{titlepage}

\section{Recursive Least Squares}

We consider the objective function

\begin{equation}
J(\theta) = 
\frac{1}{2} \sum\limits_{k=1}^N (\alpha(k) \frac{(z(k) - \theta^T \phi(k))^2}{m(k)^2}) + \frac{1}{2} (\theta - \theta_0)^T {P_0}^{-1} (\theta - \theta_0)
\end{equation}

with $P_0 > 0$ a definite matrix and $\alpha(k)$ a non-negative sequence of weighting coefficients, with $N$ measurements and $\theta_0$ an initial guess for $\theta$. We have the parametric model

\begin{equation}
z(k) = \theta^{\alpha t} (k-1) \phi (k)
\end{equation}

with $z$ and $\phi$ as measurements and $\theta$ as the parameters. We then have an estimate of the measurement vector as

\begin{equation}
\hat{z} (k) = \theta^T (k-1) \phi (k)
\end{equation}

so that the error prediction is

\begin{equation}
\varepsilon (k) = \frac{z(k) - \hat{z}(k)}{m(k)^2} = \frac{z(k) - \theta (k-1)^T \phi(k)}{m(k)^2}
\end{equation}

We can then define

\[
Z_k =
\begin{bmatrix}
\frac{z(1)}{m(1)},\frac{z(2)}{m(2)},\frac{z(3)}{m(3)},...,\frac{z(k)}{m(k)}
\end{bmatrix}^T
\]

\[
\Phi_k =
\begin{bmatrix}
\frac{\phi(1)}{m(1)},\frac{\phi(2)}{m(2)},\frac{\phi(3)}{m(3)},...,\frac{\phi(k)}{m(k)}
\end{bmatrix}^T
\]

\[
A = 
\begin{bmatrix}
\alpha(1),\alpha(2),\alpha(3),...,\alpha(k)
\end{bmatrix}
\]

and the objective function can be rewritten as

\begin{equation}
J(\theta) = \frac{1}{2} (Z_k - \Phi_k \theta)^T A (Z_k - \Phi_k \theta) + \frac{1}{2} (\theta \theta_0)^T {P_0}^{-1} (\theta - \theta_0)
\end{equation}

Then, to minimize cost, we take the gradient of $J(\theta)$ with respect to $\theta$ and set equal to $0$.

\begin{equation}
\nabla_\theta J(\theta) = - {\Phi_k}^T A Z_k + {\Phi_k}^T A \Phi_k \theta + {P_0}^{-1} \theta - {P_0}^{-1} \theta_0 = 0
\end{equation}

We then solve for $\theta$ to get

\begin{equation}
\theta = ( {\Phi_k}^T A \Phi_k + {P_0}^{-1})^{-1} ( {\Phi_k}^T A Z_k + {P_0}^{-1} \theta_0)
\end{equation}

and we define

\begin{equation}
{P_k}^{-1} =  {\Phi_k}^T A \Phi_k + {P_0}^{-1} \Rightarrow P(k)^{-1} =  {\Phi_{k-1}}^T A \Phi_{k-1} + {P_0}^{-1}
\end{equation}

then

\begin{equation}
P(k)^{-1} - P(k-1)^{-1} = \alpha (k) \frac{\phi(k)\phi(k)^T}{m(k)^2} \Rightarrow P(k)^{-1} = P(k-1)^{-1} + \alpha (k) \frac{\phi(k)\phi(k)^T0}{m(k)^2}
\end{equation}

We can then use matrix inversion lemma

\begin{equation}
(A+BC)^{-1} = A^{-1}-A^{-1}B(I+CA^{-1}B)^{-1}CA^{-1}
\end{equation}

and define $A=P(k-1)^{-1}$, $B=\alpha(k)\frac{\phi(k)}{m(k)}$, $C=\frac{\phi(k)^T}{m(k)}$ so we can then achieve

\begin{equation}
P(k) = P(k-1) - \frac{P(k-1)\alpha(k)\phi(k)\phi(k)^TP(k-1)}{m(k)^2+\alpha(k)\phi(k)^TP(k-1)\phi(k)}
\end{equation}

We then derive update law for $\theta(k)$ based on parameters in the previous iteration

\begin{equation}
\theta(k) = P(k) ( {\Phi_{k-1}}^T A Z_{k-1} + {P_0}^{-1} \theta_0 + \alpha(k) \frac{\phi(k)z(k)}{m(k)^2}) = P(k) ( {\Phi_{k-1}}^T A Z_{k-1} + {P_0}^{-1} \theta_0) + P(k) 
\alpha(k)\frac{\phi(k)z(k)}{m(k)^2}
\end{equation}

and $\theta(k-1)$ can be written as

\begin{equation}
\theta(k-1) = ( {\Phi_{k-1}}^T A \Phi_{k-1} + P_0)^{-1}( \Phi_{k-1} A Z_{k-1} + {P_0}^{-1} \theta_0)
\end{equation}

so

\begin{equation}
( {\Phi_{k-1}}^T A \Phi_{k-1} + {P_0}^{-1}) \theta(k-1) = ( {\Phi_{k-1}}^T A Z_{k-1} + {P_0}^{-1} \theta_0)
\end{equation}

Substituting this into $(14)$, we get

\[
\theta(k) = P(k) ( {\Phi_{k-1}}^T A \Phi_{k-1} + {P_0}^{-1}) \theta(k-1) + P(k) \alpha(k) \frac{\phi(k)z(k)}{m(k)^2}
\]

\begin{equation}
= P(k)P(k-1)^{-1} \theta(k-1) + P(k) \alpha(k) \frac{/phi(k)z(k)}{m(k)^2}
\end{equation}

Then, by substituting Eqn. 10 into Eqn. 16, we get

\begin{equation}
\theta(k) = P(k)(P(k)^{-1} - \alpha(k) \frac{\phi(k)\phi(k)^T}{m(k)^2} \theta(k-1) + P(k)
\end{equation}

which simplifies to

\begin{equation}
\theta(k) = \theta(k-1) + P(k) \frac{\alpha(k)\phi(k)(z(k) - \phi(k)^T\theta(k-1))}{m(k)^2}
\end{equation}

Substituting Eqn. 3 into Eqn. 18, we get final update law for unknown parameters $\theta$

\begin{equation}
\theta(k) = \theta(k-1) + P(k) \alpha(k) \phi(k) \varepsilon(k)
\end{equation}

Thus, we have derived equations for recursive least squares algorithm to minimize the cost function in Eqn. 1 to identify values of parameters in $\theta$.

\section{Least Squares and Differential Dynamic Programming}

We consider the cart-pole equation of motion

\begin{equation}
\ddot{x} = \frac{1}{m_c + m_p \sin^2 \theta} (f + m_p \sin \theta (l \dot{\theta}^2 + g \cos \theta))
\end{equation}

with $m_c$, $m_p$, and $l$ unknown, and only initial guesses for $\hat{m}_c$, $\hat{m}_p$, and $\hat{l}$ available. By parametrizing to form

\begin{equation}
\mathbf{z} = \mathbf{w}^T \phi(\mathbf{x})
\end{equation}

we result in

\begin{equation}
\underbrace{f}_{\mathbf{z}} = 
\underbrace{
\begin{bmatrix}
m_p & m_c &m_p  l
\end{bmatrix}}_{\mathbf{w}^T}
\underbrace{
\begin{bmatrix}
\ddot{x} \sin^2 \theta - g \sin \theta \cos \theta \\
\ddot{x} \\
- \sin \theta \dot{\theta}^2
\end{bmatrix}}_{\phi}
\end{equation}

Even though two equations are provided for the dynamics of the cart pole, only one is used for the parameterization because the two equations would yield the same results in the end. Thus, although they are algebraically different, the problem was simplified by using just one equation, the simpler of the two, instead of linearizing both equations to combine them for the regression.

\section{Bonus: Multidimensional Least Squares}

\end{document}